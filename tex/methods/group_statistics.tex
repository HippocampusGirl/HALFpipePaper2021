\subsection{Statistics}\label{sec:statistics}

\soft{HALFpipe} uses \soft{FSL} FMRIB Local Analysis of Mixed Effects
(FLAME) \parencite{woolrich2004} for group statistics,
because it considers the within-subject variance of lower level estimates
in its mixed effects models. In addition, its estimates are conservative,
which means they offer robust control of the false positive rate
\parencite{eklund2016}.

A common issue in fMRI studies is that the spatial extent of brain coverage
may differ between subjects. A common choice is to restrict higher-level
statistics to only those voxels that were acquired in every subject.
However, with a large variation in brain coverage, which is to be expected
when pooling multi-cohort data, sizable portions of the brain may
ultimately be excluded from analysis. To circumvent this issue,
\soft{HALFpipe} uses a re-implementation of \soft{FSL}'s \soft{flameo} in
\soft{Numpy} \parencite{harris2020}. In this implementation, a
unique design matrix is re-generated for every voxel so that only subjects
who have a measurable value for a given voxel are included. Then the model
is estimated using the FLAME algorithm. This list-wise deletion approach
depends on the assumption that voxels are missing completely at random
(MCAR), meaning that the regressors (and thus statistical values) are
independent of scanner coverage.

For group models, users can specify flexible factorial models that include
covariates and group comparisons. By default, missing values for these
variables are handled by list-wise deletion as well, but the user may
alternatively choose to replace missing values by zero in the demeaned
design matrix. The latter approach is equivalent to imputation by the
sample mean. Design matrices for the flexible factorial models are
generated using the Python module \soft{Patsy}
\parencite{smith2018}. Contrasts between groups are
specified using the \soft{lsmeans} procedure \parencite{lenth2016}.
