\subsection{Confound time series removal}

\soft{fMRIPrep} not only outputs a preprocessed image in standard space but also a spreadsheet with confound time series named \filename{confounds.tsv}. These include the (derivatives of) motion parameters (squared), aCompCor components \parencite{behzadi2007}, white matter signal, CSF signal, and global signal. A key consideration needs to be made when using \soft{fMRIPrep} confound time series in conjunction with the preprocessing steps outlined in the \hyperref[sec:preprocessing]{previous section}: Using confound time series as nuisance regressors with data that was temporally filtered or denoised can re-introduce the removed temporal or noise signals back into the voxel time series \parencite{hallquist2013}.

An example of this phenomenon may be regressing out a set of \soft{fMRIPrep} confound time series after removing low-frequency drift via temporal filtering. In practice, this means setting up a regression model for each voxel, where the voxel time series is the dependent variable and the regressors are the confound time series. The regression will yield a weight for each confound time series, so that the total model explains the maximum amount of variance (under assumption of normality). After multiplying the confound time series with these weights, their products are summed to one time series containing the confound-related signal in that voxel. This time series is then subtracted from the original voxel time series to get the result (i.e., the regression residuals). However, if the confound time series happen to contain any low-frequency drift, then their weighted sum likely will as well. It follows that subtracting a time series with temporal drift from the temporally filtered voxel data will re-introduce temporal drift, independent of whether a temporal filter was applied before.

In \soft{HALFpipe}, any (optional) denoising or filter applied to the voxel time series is also applied to the \soft{fMRIPrep} confound time series. This way, previously removed variance is not re-introduced accidentally, because it has been removed from both sides of the regression equation. For example, when the user chooses to perform \soft{ICA-AROMA} denoising, then the same denoising will be applied to the time \soft{fMRIPrep} confound time series, and the same applies when using a temporal filter. Note that this means that the confound time series generated by \soft{HALFpipe} will be different from the original \soft{fMRIPrep} confound time series, and users should take care to use the appropriate file when running custom analyses.
