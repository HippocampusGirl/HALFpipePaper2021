\subsection{Databases}

To automatically construct a neuroimaging data processing workflow, the program needs to be able to fulfill queries such as \pseudocode{``retrieve the structural image for subject x''}. Many programs implement such queries using a database system. The queries also need to flexibly interface with the logic of neuroimaging and processing pipelines, which is relevant in the context of missing scans.

In the context of missing scans, \soft{HALFpipe} always tries to execute the best possible processing pipeline based on the data that is available. For example, a field map may have been routinely acquired before each functional scan in a particular dataset. If one of these field maps is missing, \soft{HALFpipe} flexibly assigns another field map, for example one belonging to the preceding functional scan. However, \soft{HALFpipe} will not use a field map from another scan session, as field inhomogeneities are likely to have changed. Finally, \soft{HALFpipe} does not fail if a field map is missing, but simply omits the distortion correction step for that subject. Other examples include the ability of \soft{HALFpipe} to match structural to functional images, and match task events to a functional scan. This strategy is used throughout the construction of processing workflows.
