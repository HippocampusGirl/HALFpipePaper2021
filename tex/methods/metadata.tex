\subsection{Metadata}

Processing of neuroimaging data requires access to relevant metadata, such as temporal resolution, spatial resolution, and many others. Some elements of metadata, such as echo time (TE), are represented differently depending on scanner manufacturer and DICOM conversion software. The method for reading various types of data has been harmonized in \soft{HALFpipe} using the following three methods.

First, metadata can be stored in BIDS format. This means that a JavaScript Object Notation (JSON) file accompanies each image file, which contains the necessary metadata. BIDS calls this file the \term{sidecar}, and common tools such as \soft{heudiconv} \parencite{halchenko2018} or \soft{dcm2niix} \parencite{li2016} generate these files automatically. If these files are present, \soft{HALFpipe} will detect and use them. Second, instead of sidecar files, some software tools store image metadata in the NIfTI header. The NIfTI format defines fields that can fit metadata, but depending on how the image file was created, these metadata may be missing. Some conversion programs also place the metadata in the description field in free text format. This description can also be parsed and read automatically. Third, information may be incorrectly represented due to user error, incompatible units of measurement, or archaic technical considerations. In such cases, \soft{HALFpipe} provides a mechanism to override the incorrect values. For every metadata field, the user interface will prompt the user to confirm that metadata values have been read or inferred correctly. The user can choose to manually enter the correct values.
