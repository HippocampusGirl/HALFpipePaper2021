\section{Procedure}

\soft{HALFpipe} starts up as a terminal-based user interface that prompts the user with a series of questions about the dataset being analyzed and the desired analysis plan. The main stages of \soft{HALFpipe} analysis, which are detailed below, include loading data, preprocessing with \soft{fMRIPrep}, quality assessment, feature extraction, and group-level statistics. Users have the flexibility to specify the settings for each processing stage at one time or separately at each stage. If \soft{HALFpipe} is stopped and resumed at an intermediate stage, \soft{HALFpipe} will detect which stages have been completed and ask the user to indicate further analyses that are desired. For instance, the user can request preprocessing and feature extraction, but not group-level statistics, and later resume processing specifying group-level statistics only.

\subsection{Loading data}

A major advantage of \soft{HALFpipe} is that it accepts input data organized in various formats without the need for file naming conventions or a specific directory structure. Using the terminal interface, the user is asked to provide the location of the T1-weighted and fMRI BOLD image files, which are required for preprocessing, as well as field maps and task event files if available or applicable. However, \soft{HALFpipe} requires additional information linking the image files to run in an automated fashion, such as information specifying which set of images belong to the same subject.

Through the use of path templates, \soft{HALFpipe} can handle a wide range of folder structures and data layouts. The syntax for path templates is adapted from \soft{C-PAC}'s data configuration \parencite{giavasis2020b}. Instead of manually adding each input file for each subject separately, as is done in the \soft{SPM} or \soft{FSL} user interfaces, the template describes the pattern used for naming files. That pattern can match many file names, thereby reducing the amount of manual work for the user. For example, when placing the tag \filename{\{subject\}} in the file path \filename{\{subject\}\_t1.nii.gz}, all files of which the name ends in \filename{\_t1} and have the extension \filename{.nii.gz} will be selected. The part of the filename that comes before \filename{\_t1} is now interpreted by the parsing algorithm as the subject identifier. When multiple files from different modalities have the same subject identifier, or session number, etc., they will be matched automatically by these tags. Automated processing workflows can then be constructed around the resulting data structure.

\begin{tablebox}[label={table:patterns}]{Examples of path template syntax}

\begingroup%
\fboxsep=1pt%
\newcolumntype{L}{>{\raggedright\arraybackslash}X}%
\newcolumntype{K}[1]{>{\raggedright\arraybackslash}m{#1}}%
\renewcommand{\arraystretch}{1.35}%
\begin{tabularx}{\textwidth}{@{} | K{2cm} |
L | L | @{}} 
\hhline{~|-|-|}
\multicolumn{1}{c|}{} & 
\textbf{Example 1} & 
\textbf{Example 2} \\
\hhline{-|-|-|}
Path template &%
\texttt{data/\colorbox{leaLightPurple}{\{subject\}}/bold\_rest.nii.gz} &%
\texttt{data/subject\colorbox{leaLightPurple}{\{subject\}}/bold\_\colorbox{leaLightBlue}{\{task\}}.nii.gz} \\
\hhline{-|-|-|}
Matches these files &%
\makecell[l]{%
\texttt{data/\colorbox{leaLightPurple}{subject01}/bold\_rest.nii.gz}\\%
\texttt{data/\colorbox{leaLightPurple}{subject02}/bold\_rest.nii.gz}\\%
\texttt{data/\colorbox{leaLightPurple}{subject03}/bold\_rest.nii.gz}\\%
\texttt{data/\colorbox{leaLightPurple}{phantom}/bold\_rest.nii.gz}} &%
\makecell[l]{%
\texttt{data/subject\colorbox{leaLightPurple}{01}/bold\_\colorbox{leaLightBlue}{rest}.nii.gz}\\%
\texttt{data/subject\colorbox{leaLightPurple}{02}/bold\_\colorbox{leaLightBlue}{rest}.nii.gz}\\%
\texttt{data/subject\colorbox{leaLightPurple}{03}/bold\_\colorbox{leaLightBlue}{rest}.nii.gz}\\%
\texttt{data/subject\colorbox{leaLightPurple}{01}/bold\_\colorbox{leaLightBlue}{task}.nii.gz}} \\
\hhline{-|-|-|}
Does not match &%
\makecell[l]{%
\texttt{data/subject01/bold\_\colorbox{leaLightGrey}{task}.nii.gz}\\%
\texttt{data/\colorbox{leaLightGrey}{subfolder}/subject01/bold\_rest.nii.gz}} &%
\makecell[l]{%
\texttt{data/\colorbox{leaLightGrey}{phantom}/bold\_rest.nii.gz}\\%
\texttt{data/\colorbox{leaLightGrey}{subfolder}/subject01/bold\_rest.nii.gz}} \\
\hhline{-|-|-|}
\end{tabularx}\par
\vspace*{2mm}
Note: The grey highlight shows the part of the file name that is
responsible for not matching.
\endgroup

\end{tablebox}


In contrast to \soft{C-PAC}'s data configuration syntax, \soft{HALFpipe} path templates use BIDS tags \parencite{gorgolewski2016b}. \soft{HALFpipe} path templates can be further specified by adding a colon and a regular expression after the tag name (as in standard Python regular expression syntax). For example, \filename{\{subject:[0-9]\*\}} will only match subject identifiers that contain only digits. This can be useful for more complex data layouts, such as when multiple datasets are placed in the same directory, and only a single subset is to be used. For more examples, see Table~\ref{table:patterns}.

In the \soft{HALFpipe} user interface, the user receives feedback on how many and which files are matched, so that the path templates can be entered interactively. Importantly, after finishing the configuration process via the user interface, all files are internally converted into the standardized BIDS structure, which is a prerequisite for running \soft{fMRIPrep}. However, no copies of files are made, the conversion is based entirely on symbolic links (aliases) to the original files. If the data are already in BIDS format, \soft{HALFpipe} will still carry out this conversion for consistency. The resulting dataset in BIDS format is then stored in the working directory in a subfolder called \filename{rawdata}.

\subsection{Quality assessment}

Quality assessment can be performed in an interactive, browser-based user interface (see \autoref{fig:qa}). \soft{HALFpipe} provides a detailed user manual for quality assessment that is linked on the web page. The web app shows report images of several preprocessing steps such as T1 skull stripping and normalization, BOLD tSNR, motion confounds, ICA-based artifact removal, and spatial normalization (see the methods section on~\nameref{sec:qamethods}). These images can be visually inspected and rated by the viewer as either good, uncertain, or bad.

Ratings will be saved in the local browser storage. Once completed, they can be downloaded in JSON format to be read by \soft{HALFpipe}. If placed in the working directory, ratings will be automatically detected by \soft{HALFpipe} and used to exclude subjects for group-level statistics. Additionally, \soft{HALFpipe} will automatically detect all other JSON files whose names start with \filename{exclude}, to accommodate quality assessment by multiple researchers. In the case of conflicts between ratings, the lower rating will be used.

\soft{HALFpipe} will include as much data as possible while excluding all scans rated as ``bad''. Ratings of ``good'' and ``uncertain'' will be included for group analysis. A ``bad'' rating for any report image related to structural/anatomical processing will exclude the entire subject. A ``bad'' rating for any report image related to functional image processing will only exclude the specific functional scan. This means that if a subject has one ``bad'' scan, its other scans may still be included for group statistics.

In addition, the mean framewise displacement, percentage of frames with a framewise displacement above a specified threshold, percentage of the independent components that were classified as noise, and mean gray matter tSNR from all subjects is displayed in box plots. Next to the report images, links to the source images are shown so that these can be inspected in more detail by opening them in a preferred image viewer (e.g., \soft{fsleyes}).

Images can be zoomed by clicking them. For faster operation by advanced users, rating and navigation are accessible not just via user interface buttons, but also via keyboard shortcuts based on the WASD keys. Pressing the A goes back one image and D goes ahead, whereas W, S and X rate an image as good, uncertain or bad, respectively. The web app offers an overview chart that indicates subjects preprocessed successfully and subjects with errors, a chart with quality ratings, and box plots reflecting the sample distributions for motion, noise components, and temporal signal-to-noise ratio (tSNR). All three are implemented so that users can hover over chart elements with their cursor to view meta-information, such as the subject identifier, and click to navigate to the associated report images. The HTML file is built as a frameworkless web app using TypeScript. Source code is available at \url{https://github.com/HALFpipe/QualityCheck}.

\soft{HALFpipe} shows two report images for each subject on structural/anatomical processing and four additional images for each type of functional scan. Detailed explanations may be found in the quality assessment manual at \url{https://github.com/HALFpipe/HALFpipe#quality-checks}.

\begin{enumerate}

\item

T1w skull stripping shows the bias-field corrected anatomical image overlaid with a red line that outlines the brain mask. The user must check that no brain regions are missing from the mask, and that portions of the skull or head are not included in the mask.

\item

T1w spatial normalization shows the anatomical image resampled to standard space overlaid with a brain atlas in standard space. The user needs to check whether the regions of the atlas closely match the resampled image.

\item

Echo planar imaging (EPI) tSNR shows the temporal signal-to-noise ratio of the functional image after preprocessing using \soft{fMRIPrep}. The user must check that signal recovery is distributed uniformly throughout the brain, and exclude scans with asymmetry, distortions, localized signal drop-out, or striping artifacts.

\item

EPI Confounds shows the carpet plot \parencite{power2017,aquino2020} generated by \soft{fMRIPrep}. A carpet plot is a two-dimensional plot of time series within a scan, with time on the \emph{x}-axis and voxels on the \emph{y}-axis. Voxels are grouped into cortical gray matter (blue), subcortical gray matter (orange), cerebellum (green), and white matter and cerebrospinal fluid (red). Above the carpet plot are time courses (\emph{x}-axis) of the magnitude (\emph{y}-axis) of framewise displacement (FD), global signal (GS), global signal in CSF (GSCSF), global signal in white matter (GSWM), and DVARS, which is the temporal change in root-mean-square intensity (D being the temporal derivative of time courses and VARS the root-mean-square variance over voxels). The user must look for changes in heatmap/intensity in relation to motion and signal changes above. Abrupt changes in the carpet plot may correspond to motion spikes, whereas extended signal changes may indicate acquisition artifacts caused by defective scanner hardware.

\item

EPI ICA-based artifact removal shows the time course of the mean signal extracted from each ICA-component and its classification as either signal (green) or noise (red). This figure is generated by \soft{fMRIPrep}. For each component, there is a spatial map (left), the time series (top right) and the power spectrum (bottom right). The user must check that components classified as noise do not contain brain networks or temporal patterns that are known to be signal.

\item

EPI spatial normalization shows the functional image after preprocessing using \soft{fMRIPrep} overlaid with a brain atlas in standard space. As before, the user must check whether the regions of the atlas closely match the resampled image.

\end{enumerate}

\begin{figure}[!tb]
    \begin{adjustwidth}{-1cm}{}
        \hsize=\linewidth%
        \includegraphics[width=\linewidth]{./fig/quality-assessment/quality-assessment-crop.pdf}
        \caption{\textbf{Quality assessment user interface.} The top panel shows the charts view, containing one chart for processing status, one for quality ratings and one for image quality metrics. In the top left corner, the navigation menu is open, which shows the option to export ratings for use in group statistics. The bottom panel contains a screenshot of the explorer view that allows the user to navigate across subjects and image types. The explorer view shows the currently selected report image on the right, along with its rating, related images, and the source files that were used to construct it. By clicking on the image, or selecting the report detail view in the navigation menu, the image can be zoomed and panned using the mouse.}\label{fig:qa}
    \end{adjustwidth}
\end{figure}

\subsection{Feature extraction}\label{sec:featureextraction}

Following preprocessing, \soft{HALFpipe} can extract several \term{features} that are commonly used in resting-state and task-based analysis. These include various ways of examining functional connectivity between brain regions (seed-based connectivity, network-template (or dual) regression, atlas-based connectivity matrices), as well as measures of local activity (ReHo, fALFF). \soft{HALFpipe} allows the user to choose several region-of-interest masks (seeds), template networks, and atlases, for which a threshold indicates the minimum overlap the user requires between seeds, template networks, or atlas regions and the subjects' fMRI data. For each feature, the user can change the default settings for spatial smoothing and temporal filtering, and choose the confounds to be removed. The user is offered the option to extract the same feature multiple times, each time varying the preprocessing, confound, and denoising settings to explore the impact of analytical decisions in a \term{multiverse analysis}. Of note, for selected features some options are not available. For example, spatial smoothing is disabled for atlas-based connectivity matrices \parencite{alakorkko2017}, or performed \emph{after} ReHo and fALFF have been calculated (see Table~\ref{table:settings}).

A brief description of the features is provided in Box~\ref{box:features}.

\subsection{Group-level statistics}

Group-level statistics on individual features can be performed with \soft{FSL}'s FLAME algorithm. Subjects who had poor quality data in the interactive quality assessment are excluded. In addition, subjects can be excluded based on movement by selecting the maximum allowed mean framewise displacement (FD) and percentage of outlier frames (i.e., frames with motion higher than the specified FD threshold).

For group-level statistics, users can choose to calculate the intercept only (i.e., mean across all subjects) or run flexible factorial models. For the latter, \soft{HALFpipe} prompts the user to specify the path to a covariates file (multiple file formats are supported) containing subject IDs, group membership, and other variables, and to specify whether these are continuous or categorical. Missing values in the covariates file can be handled with either listwise deletion or mean substitution. The user can specify main effects and interactions between variables, while within-group regressions against a continuous variable (e.g., symptom severity) is also possible.

\begin{featurebox}[label={box:features}]{Overview of HALFpipe features}

\paragraph{Task-based activations}

A first-level general linear model (GLM) is run for event-related or block designs. GLM regressors describing the stimulus presentations for each of the task conditions are convolved with a double Gamma HRF and the overall model is fit for each voxel in the brain using \soft{FSL FILM} \parencite{woolrich2001}. Contrasts of interest are tested, which results in a whole-brain task activation map for comparisons between task conditions.

\paragraph{Seed-based connectivity}

Average BOLD time series are extracted from a seed region of interest (ROI), which is defined by a binary mask image. This time series is used as a regressor in a first-level GLM, where the model is fit for each voxel in the brain using \soft{fsl\_glm}. This results in a whole-brain functional connectivity map that represents the connectivity strength between the ROI and each voxel in the brain.

\paragraph{Network-template (or dual) regression}

Subject-specific representations of connectivity networks (e.g., default mode, salience, task-positive networks) are generated using dual regression \parencite{beckmann2009} with \soft{fsl\_glm}. In a first regression model, the set of network template maps is regressed against the individual fMRI data, which generates time series for each of the template networks. Next, a second regression model is run, regressing the network time series against the individual fMRI data. This generates subject-specific spatial representations of each of the template networks, which can be considered to represent the voxelwise connectivity strength within each of the networks.

\paragraph{Atlas-based connectivity matrix}

Average time series are extracted from each region of a brain atlas of choice using custom code inspired by \soft{Pypes} \parencite{savio2017} and \soft{Nilearn} \parencite{abraham2014}. From these, a pairwise connectivity matrix between atlas regions is calculated using Pearson product-moment correlations using \soft{Pandas} \parencite{mckinney2010}, which represent the pairwise functional connectivity between all pairs of regions included in the atlas.

\paragraph{Regional homogeneity (ReHo)}

Local similarity (or synchronization) between the time series of a given voxel and its nearest neighboring voxels is calculated using Kendall's coefficient of concordance \parencite{zang2004} using \soft{FATCAT}'s \soft{3dReHo} which is distributed with \soft{AFNI} \parencite{taylor2013}.

\paragraph{Fractional amplitude of low frequency fluctuations (fALFF)}

Variance in amplitude of low frequencies in the BOLD signal is calculated, dividing the power in the low frequency range (0.01--0.1 Hz) by the power in the entire frequency range \parencite{zou2008} with a customized version of the \soft{C-PAC} implementation of fALFF.\@

\end{featurebox}
